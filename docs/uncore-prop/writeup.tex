%%MaD.tex - Notes taken for Materials and Devices Lecture
%%Author: Andy Goetz
%%Date Modified: 10-7-09
%%License: Ask me before reproducing/modifying, etc.


\documentclass{article}

%Make sure you have the file ShumanNote.scy in the same directory as
%this one. It has contains the style sheet for ECE111, and is needed
%to standardize the layout of LateX documents created for the class.
\usepackage{ShumanNotes} 
\usepackage{algorithm}
\usepackage{algpseudocode}
\usepackage{tikz}
\usepackage{array}
\usepackage{program}
\usepackage{listings}
\pdfpagewidth 8.5in 
\pdfpageheight 11in
\usepackage{titlesec}
\usepackage{ucs}
\usepackage[utf8x]{inputenc}
%% \titleformat{\section}{\bfseries}{\thesection}{1em}{}
%% \titleformat{\subsection}{\bfseries}{\thesubsection}{1em}{}
%This package is used to line up pictures 
\usepackage{graphicx}
\usepackage{fancyvrb}
\usepackage{listings}
%allows cursive font
%\usepackage{amsmath}

%allows hyperlinks 
\usepackage{hyperref}

\newcommand{\HRule}{\rule{\linewidth}{0.5mm}} 
%% \renewcommand\thesubsection{\alph{subsection})}
%% \renewcommand\thesection{\arabic{section})}
\lhead{Homework 1}
\begin{document}

%% These commands allow me to use cursive letter for things such as
%% length.  Note that on ubuntu linux, this required installation of
%% the package 'texlive-fonts-extra'. 
%% Taken from
%% http://www.latex-community.org/forum/viewtopic.php?f=5&t=1404&start=0
\newenvironment{frcseries}{\fontfamily{frc}\selectfont}{}
\newcommand{\textfrc}[1]{{\frcseries#1}}
\newcommand{\mathfrc}[1]{\text{\textfrc{#1}}}



\begin{titlepage}
 
\begin{center}
 
 
\textsc{\LARGE ECE 510 System Verilog}\\[1.5cm]
 
\textsc{\Large Portland State University}\\[0.5cm]
 
 
% Title
\HRule \\[0.4cm] { \huge \bfseries Homework 1}\\[0.4cm]
 
\HRule \\[1.5cm]
 
% Author and supervisor
\begin{minipage}{0.4\textwidth}
\begin{center} \large
Andy \textsc{Goetz}\\
\end{center}
\end{minipage}

\vspace{1cm}
%\includegraphics[height=4.75in]{sh_landing_gear.png}
 
 
\end{center}
\vfill
\begin{center}
% Bottom of the page
{\large \today}

\end{center} 
\end{titlepage}

\newpage
\thispagestyle{empty}
\vspace*{0.6\paperheight}
\begin{center}\textit{This page intentionally left blank.}\end{center}

\newpage
\setcounter{page}{1}

\section{Overview}
This project consists of a finite state machine to control the landing
gear on an airplane. The state transition diagram can be seen in
figure \ref{std}. A key for the transition logic can be seen in table
\ref{key}, and the per-state outputs can be seen in figure
\ref{outputs}.


\end{document}

